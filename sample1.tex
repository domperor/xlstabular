\documentclass[a4paper,10pt,twoside,uplatex]{jsarticle}
\usepackage{xlstabular}
\begin{document}
\begin{itemize}
\item 最初の例\ \ \ \ 
\xlstabularstart[kaigyouhaba=0.5zw,midasiyoko=true,midasi=left,nokori=right]{
表は	この	空間に
Tabキー	で	区切って
記述	すること	ができます
Excel	から直で	コピペできる!
}\xlstabularend

\ 


\item 2個目の例

\ 

\xlstabularstart{
最後のコマが空欄なので	\{\}(ブレースの組)を挟みます→	{}
途中のコマが空欄なので→		←\{\}(ブレースの組)は不要です
最後のコマにコントロール・スペース	を入れたので\{\}(ブレースの組)を	挟みます→\ {}
}\xlstabularend

\ 

\item カテゴリコードの表

\ 

\xlstabularstart[kaigyouhaba=0.2zw]{
カテゴリコード	役割	通常このカテゴリコードを持つ文字
0	制御綴の開始	\textbackslash
1	グループの開始	\{
2	グループの終了	\}
3	数式モード移行	\$
4	アラインメントタブ	\&
5	行の終了	\textasciicircum\textasciicircum M
6	パラメタ文字	\#
7	上付き文字	\textasciicircum
8	下付き文字	\_
9	無視される文字	\textasciicircum\textasciicircum @
10	空白文字	(半角スペース),\textasciicircum\textasciicircum I
11	英文字	A〜Z,a〜z
12	記号	0〜9,!,@など
13	アクティブ文字	\textasciitilde
14	コメント開始	\%
15	存在即エラー	\textasciicircum\textasciicircum ?
}\xlstabularend
\end{itemize}

\end{document}